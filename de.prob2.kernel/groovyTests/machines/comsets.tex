\documentclass{article}
\usepackage{fuzz}

\begin{document}

\subsection*{Test framework}
We alternately execute an operation and check its outcome, so we have to
phases $op$ and $check$. Additionally we have a counter to specify wich testcase
should be executed.
\begin{zed}
  Phase ::= op | check \\
  Testframe \defs [~test:\nat; phase : Phase~] \\
  TestframeInit \defs [~Testframe | test=1 \land phase=op~] \\
\end{zed}
We define basic schemas for operations and checks, checks do not change the state.
\begin{zed}
  Op \defs [~\Delta State;\Delta Testframe | phase\mapsto phase' = op\mapsto check  \land test=test'~]\\
  Check \defs [~\Xi State;\Delta Testframe | phase\mapsto phase' = check\mapsto op  \land test'=test+1~]\\
\end{zed}

\subsection*{State and initialisation}
We need two types
\begin{zed}
  A ::= a1 | a2 | a3 \\
  B ::= b1 | b2 | b3 \\
\end{zed}
Our state consists of a set of tuples, which is initialised empty
\begin{zed}
  State \defs [~s : A \rel B~]\\
  StateInit \defs [State | s = \emptyset~]\\
\end{zed}
Now we combine the test frame and the state itself:
\begin{zed}
  FullState \defs Testframe \land State \\
  Init \defs [~FullState' | TestframeInit' \land StateInit' ~]\\
\end{zed}

\subsection*{Comprehension sets with several parameters}
The first test checks if a comprehension set whose arguments are not
ordered alphabetically, is computed correctly
\begin{zed}
  Test1 \defs [~Testframe | test=1~]\\
  Op1 \defs [~Test1;Op | s' = \{~ y:A; x:B | y=a1 \land x=b1 ~\}~]\\
  Check1 \defs [~Test1;Check | s = \{a1 \mapsto b1\}~]\\
\end{zed}
The second test is the same, but with alphabetically ordered arguments
\begin{zed}
  Test2 \defs [~Testframe | test=2~]\\
  Op2 \defs [~Test2;Op | s' = \{~ x:A; y:B | x=a2 \land y=b2 ~\}~]\\
  Check2 \defs [~Test2;Check | s = \{a2 \mapsto b2\}~]\\
\end{zed}


\end{document}
